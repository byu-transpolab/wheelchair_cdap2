\documentclass[3p, authoryear, review]{elsarticle} %review=doublespace preprint=single 5p=2 column
%%% Begin My package additions %%%%%%%%%%%%%%%%%%%
\usepackage[hyphens]{url}

  \journal{Submitted to Transportation Research Part A} % Sets Journal name


\usepackage{lineno} % add
\providecommand{\tightlist}{%
  \setlength{\itemsep}{0pt}\setlength{\parskip}{0pt}}

\usepackage{graphicx}
%%%%%%%%%%%%%%%% end my additions to header

\usepackage[T1]{fontenc}
\usepackage{lmodern}
\usepackage{amssymb,amsmath}
\usepackage{ifxetex,ifluatex}
\usepackage{fixltx2e} % provides \textsubscript
% use upquote if available, for straight quotes in verbatim environments
\IfFileExists{upquote.sty}{\usepackage{upquote}}{}
\ifnum 0\ifxetex 1\fi\ifluatex 1\fi=0 % if pdftex
  \usepackage[utf8]{inputenc}
\else % if luatex or xelatex
  \usepackage{fontspec}
  \ifxetex
    \usepackage{xltxtra,xunicode}
  \fi
  \defaultfontfeatures{Mapping=tex-text,Scale=MatchLowercase}
  \newcommand{\euro}{€}
\fi
% use microtype if available
\IfFileExists{microtype.sty}{\usepackage{microtype}}{}
\usepackage{natbib}
\bibliographystyle{apalike}
\usepackage{longtable,booktabs,array}
\usepackage{calc} % for calculating minipage widths
% Correct order of tables after \paragraph or \subparagraph
\usepackage{etoolbox}
\makeatletter
\patchcmd\longtable{\par}{\if@noskipsec\mbox{}\fi\par}{}{}
\makeatother
% Allow footnotes in longtable head/foot
\IfFileExists{footnotehyper.sty}{\usepackage{footnotehyper}}{\usepackage{footnote}}
\makesavenoteenv{longtable}
\ifxetex
  \usepackage[setpagesize=false, % page size defined by xetex
              unicode=false, % unicode breaks when used with xetex
              xetex]{hyperref}
\else
  \usepackage[unicode=true]{hyperref}
\fi
\hypersetup{breaklinks=true,
            bookmarks=true,
            pdfauthor={},
            pdftitle={Coordinated daily activity patterns of wheelchair users.},
            colorlinks=false,
            urlcolor=blue,
            linkcolor=magenta,
            pdfborder={0 0 0}}
\urlstyle{same}  % don't use monospace font for urls

\setcounter{secnumdepth}{5}
% Pandoc toggle for numbering sections (defaults to be off)

% Pandoc citation processing

% Pandoc header
\usepackage{booktabs}
\usepackage{booktabs}
\usepackage{longtable}
\usepackage{array}
\usepackage{multirow}
\usepackage{wrapfig}
\usepackage{float}
\usepackage{colortbl}
\usepackage{pdflscape}
\usepackage{tabu}
\usepackage{threeparttable}
\usepackage{threeparttablex}
\usepackage[normalem]{ulem}
\usepackage{makecell}
\usepackage{xcolor}
\usepackage{siunitx}
\newcolumntype{d}{S[input-symbols = ()]}



\begin{document}
\begin{frontmatter}

  \title{Coordinated daily activity patterns of wheelchair users.}
    \author[Brigham Young University]{Gregory Macfarlane\corref{1}}
   \ead{gregmacfarlane@byu.edu} 
    \author[Brigham Young University]{Nate Lant}
   \ead{natelant@gmail.com} 
      \address[Brigham Young University]{Civil and Environmental Engineering Department, 430 Engineering Building, Provo, Utah 84602}
      \cortext[1]{Corresponding Author}
  
  \begin{abstract}
  Individuals who use wheelchairs or who have other mobility challenges are sometimes unable to effectively access modern mobility as a service (MaaS) systems including application-based ridehailing and on-demand microtransit, etc. Even designing a MaaS targeted at these users is challenging, given the limited prior analysis of their travel and activity patterns. In this research, we present an initial attempt to model the daily activity pattern choice of respondents to the 2017 National Household Travel Survey who self-identify as using a wheelchair. We find that wheelchair use is a significant predictor of activity patterns, with individuals who use wheelchairs considerably less likely to choose out-of-home daily patterns. We additionally find that including wheelchair use as a variable in a model with a common person-type segmentation is statistically preferable to specifying wheelchair users as an independent segment. Further analysis of household-level coordination of daily activity patterns using more detailed surveys is warranted.
  \end{abstract}
  
 \end{frontmatter}

\hypertarget{intro}{%
\section{Introduction}\label{intro}}

In 1990, the United States Congress passed the Americans with Disabilities Act (ADA),
seeking to protect individuals with qualifying disabilities from discrimination
in seeking employment (Title I), while using public services including
transportation systems (Title II), and in public accommodations (Title III) among
other specifics.

The accommodation that has arguably caused the most challenges
for many transportation service providers has been ensuring equal
access for individuals who use wheelchairs. Buses and trains had to be reengineered
with low floors and access ramps; elevators and ramps needed to be installed
in stations alongside escalators and stairs; and many traditional automobiles
remain inaccessible --- or at least without substantial modification --- to
wheelchair users. This last challenge is a particular concern for transportation
network companies (TNC's), who often use private vehicles owned by individual
operators.

Though the law only requires agencies to provide reasonable accommodation on
public conveyances and
does not try to establish equality in outcomes, the passage of 30 years provides a
convenient time to consider what gaps and challenges persist for wheelchair
users in accessing and using the transportation system. Specifically, what gap
exists in the observed travel behavior outcomes of wheelchair users vis a vis
the non wheelchair using population, all else equal? And more importantly, how
should this gap be applied within travel forecasting models and related planning
activities?

In this paper, we present an analysis of individual daily activity pattern choice
for a sample of respondents to the 2017 National Household Travel Survey (CITE),
incorporating the individual's wheelchair use as an explanatory variable.
We then apply the estimates obtained from the choice analysis in a modified
activity-based model for the Wasatch Front (Salt Lake City and outlying areas)
metropolitan region in Utah.

The paper proceeds in a typical fashion. A \protect\hyperlink{literature}{literature review}
discusses prior attempts to evaluated and quantify the travel behavior of
users with disabilities. A section \protect\hyperlink{methodology}{describing the methodology}
of the choice analysis and model application is followed by \protect\hyperlink{results}{a discussion of the results}
from both analyses. The paper concludes with \protect\hyperlink{discussion}{a discussion}
of limitations in this analysis and associated avenues for future research and
policy intervention.

\hypertarget{sec-literature}{%
\section{Literature}\label{sec-literature}}

The question of how the travel behavior of individuals with disabilities differs
from the travel behavior of those without disabilities has been addressed
previously in a number of studies. That said, the literature is often confusing
due to the wide variety of specific disabilities people may have: wheelchair use,
other ambulatory disabilities, visual impairment, and a multitude of other
conditions are each likely to affect an individual's need for accommodation in
travel --- and perhaps consequently their travel behavior --- in a number of ways.
A number of recent studies have also looked at the role of developmental \citet{Wasfi2007}
and intellectual \citet{Feeley2019} disabilities on travel patterns. However,
given the motivations of this research laid out in the introduction, we will attempt to highlight
research on the mobility and activity patterns of wheelchair users, though
drawing specific distinctions is not always possible.

Among the first considerations is research attempting to quantify the size
of the disabled or wheelchair using population. Data from the 2017 NHTS
suggests that there are 13.4 million individuals with travel-limiting disabilities
in the U.S., of whom 20 percent (or 2.7 million) use wheelchairs \citet{Brumbaugh2018}.
Another analysis of Census data suggests that 37 percent of individuals over
age 65 with disabilities use wheelchairs; both the number of wheelchair users
and the number of total individuals with disabilities are likely to rise
in the coming years with the relative aging of the U.S. population \citet{Sweeney2004}, Laplante2003.

With regards to travel patterns, surveys of the general population and surveys
specifically targeted at individuals with disabilities both reveal significant and
meaningful differences compared to individuals without disabilities. Specific
findings include that individuals with disabilities leave their homes on fewer
days if ever \citet{Sweeney2004}, make fewer daily trips
\citet{Schmocker2005}, Brumbaugh2018, make fewer work trips and more
healthcare maintenance trips \citet{Ermagun2016}, rely more on others for their travel \citet{Sweeney2004},
and have considerably restricted mode
choices \citet{Rosenbloom2007}, Ruvolo2020. These differences in mobility and activity patterns
have important and observed negative implications for the individual's access to
opportunity for employment \citet{Rosenbloom2007}, Lubin2012 and social interaction
\citet{Bascom2017}, Velho2016.

The underlying reasons why wheelchair users exhibit different travel patterns
than other individuals are varied and include both technical and attitudinal
barriers. Technical barriers include poor access to private vehicles
\citet{VanRoosmalen2010}, poorly maintained sidewalk and pedestrian infrastructure
\citet{frackelton2013measuring}, lack of physical access to TNC
vehicles \citet{Ruvolo2020}, bus ramp complication and malfunction \citet{Velho2016},
and numerous other problems across many modes.
Attitudinal barriers include feeling shame when safety alarms alert all
passengers that a ramp is being deployed or motion sickness when they are forced
to travel backwards in the wheelchair priority area \citet{Velho2016}, or
suffering outright discrimination from TNC operators \citet{Bascom2017}.

In spite of this relatively mature literature, however, there has not to
date been a rigorous evaluation of the travel behavior of individuals with
disabilities within the framework of a travel activity model. As a result,
recent attempts to simulate or model services aimed at this population have
needed to make simplifying assumptions. In an attempt to model demand for a MaaS
system targeted at wheelchair users in
Berlin, Bischoff \citet{Bischoff2019} simply assumed demand for this service
would be similar to current demand for the regional paratransit system, augmented
by a mode shift from taxi. In this initial study there was no link between the
MaaS trips and the daily activities of the wheelchair users; there was not even a
good understanding of likely trip origin, destination, or length distribution.

Including wheelchair users or wheelchair use status in a regularized travel model
framework would help to fill two important gaps in the current literature. First,
the comparison would help to illuminate the travel behavior characteristics of
this important population within a framework that is readily understandable
\textbackslash emph\{vis a vis other population segments. Second, researchers engaged in
policy and planning work for this population could replace simplifying assumptions
with plausible daily activity patterns rooted in observed behavior.

\hypertarget{methodology}{%
\section{Methods}\label{methodology}}

Activity-based models are a relatively mature construct in travel behavior
research and in practical demand forecasting \citet{rasouli2014activity}. Activity-based models attempt
to recreate the long- and short-term decision patterns of individuals within a
chain of econometric and statistical choice models. The specific sub-models included
in this chain can vary between specific implementations, but a recent
open-source project --- ActivitySim \citet{activitysim} --- implements a popular
set of models \citet{davidson2010ct}. Specifically, the ActivitySim demonstration
model is a part of the ``Travel Model One'' model for the Metropolitan
Transportation Commission (MTC, San Francisco Bay) \citet{erhardt2012mtc}.
For simplicity and comparison with other models, we apply the ActivitySim model
in this research.

The first model in the ActivitySim model chain is a \emph{daily activity pattern}
of the type described by Bradley and Vovsha \citet{Bradley2005}. This model
allows individuals to choose one of three daily activity patterns:

\begin{itemize}
\tightlist
\item
  Mandatory (\(M\)) daily patterns revolve around school and work activities that
  are typically considered non-discretionary. These activities and the travel
  to them anchor an individual's daily schedule, though other tours are possible.
\item
  Non-Mandatory (\(NM\)) daily patterns involve only discretionary activities:
  shopping, maintenance, etc.
\item
  At-Home (\(H\)) daily patterns describe the schedule and activities of
  individuals who never leave the home during the travel day.
\end{itemize}

The choice between the daily patterns is described with a multinomial logit
model \citet{Domencich1975}, where the utility functions for each option are
determined by an individual's socioeconomic characteristics and person type
segment. The specific innovation of the Bradley and Vovsha model
\citet{Bradley2005} is that the daily activity patterns are coordinated, or that
the choice of one individual in a household influences the choice probability of
other household members. We leave household-level coordination to future
research and instead examine individual uncoordinated choice utilities.

We do, however, adopt the person type segmentation strategy employed by
ActivitySim; segmentation allows for heterogeneity in available
alternatives and utility coefficients between individuals with highly divergent
expected behaviors. For example, full time workers and pre-driving age school children
will have strongly different responses to income, automobile availability, and
and other variables in determining their most likely daily pattern.
A complete descriptive list of each person type is given in the data
section below.

\hypertarget{data}{%
\subsection{Data}\label{data}}

Data for this study comes from the 2017 NHTS \citet{nhts2017}. We restrict the
data to households where the metropolitan statistical area (MSA) population size
is between one and three million people. There are 76,367 individuals in 36,497
households that responded to the NHTS from these areas, though not all of these
records are useful due to missing or incomplete data in key variables.

The NHTS releases public data in separate tables for persons, households,
trips and vehicles; to determine the daily activity pattern for a given
individual it was necessary to transform the trips table into a table of
activities. We did this by reconstructing a schedule for each person from the
reported trip origin and destination activity codes. We then determined whether
each reported tour (a chain of activities away from the individual's home)
contained a mandatory school or work activity. If any tour contained a mandatory
activity, the person's entire daily activity pattern was classified as ``mandatory'';
if not, the daily activity pattern was ``non-mandatory.'' By identifying respondents
in the persons table without records in the trips table, we can determine
individuals with a ``home'' daily activity pattern.

ActivitySim classifies persons into seven person segments, though we only
consider four types in this study, defined as follows:

\begin{itemize}
\tightlist
\item
  Full-time workers (FW) - reported working ``full-time'' at their primary job.
\item
  Part-time worker (PW) - reported working ``part-time'' at their primary job,
  as well as any person who reported being a ``non-worker'' or ``retired'' who nevertheless
  reported a work or school activity.
\item
  Non-working adults (NW) - reported ``unemployed'' as their primary activity
  of the previous week, as well as individuals over 18 who were not classified
  elsewhere.
\item
  Retired (RT) - reported ``retired'' as their primary activity of the previous
  week, or who are over the age of 65 and reported that they were not workers.
\end{itemize}

The other three person types are university students, schoolchildren under
driving age, and driving-age schoolchildren. A limited number of individuals who
could plausibly be considered university students responded to the NHTS, so we
cannot estimate reliable choice models. Among schoolchildren of any age, too few
report using wheelchairs to justify including these segments in this study.

The NHTS has a number of questions where respondents can indicate a disability
for themselves or other household members. Each respondent is asked ``Do you
have a condition or handicap that makes it difficult to travel outside of the
home?'' If the answer is yes, several follow-up questions are asked, including
``Do you use any of the following medical devices? Select all that apply.'' The
list of medical devices respondents can indicated includes canes, walkers,
seeing-eye dogs, crutches, motorized scooters, manual wheelchairs,
motorized wheelchairs, or something else (other).
For this study, we identify wheelchair users as respondents who report using a
manual wheelchair, mechanical wheelchair, or motorized scooter.

The specific variables included in the daily activity pattern choice models
are based initially on the variables used in the ActivitySim example model (which are
those used in MTC Travel Model One. The variables available include the age of
the person and the household income treated as categorical ranges; gender, work,
and college degree status are treated as binary values. Automobile availability is
included via a binary ``sufficiency'' variable where a household with at least as
many vehicles as adults is considered ``auto sufficient.'' Descriptive statistics
of the model variable within each person segment are given in
Table \ref{tab:descriptive-stats}

\begin{landscape}\begin{table}

\caption{\label{tab:descriptive-stats}Model Estimation Data: Descriptive Statistics}
\centering
\resizebox{\linewidth}{!}{
\begin{tabular}[t]{llrrrrrrrr}
\toprule
\multicolumn{2}{c}{ } & \multicolumn{2}{c}{Full-time worker (N=16188)} & \multicolumn{2}{c}{Non-worker (N=3723)} & \multicolumn{2}{c}{Part-time worker (N=4028)} & \multicolumn{2}{c}{Retired (N=10060)} \\
\cmidrule(l{3pt}r{3pt}){3-4} \cmidrule(l{3pt}r{3pt}){5-6} \cmidrule(l{3pt}r{3pt}){7-8} \cmidrule(l{3pt}r{3pt}){9-10}
  &    & Mean & Std. Dev. & Mean  & Std. Dev.  & Mean   & Std. Dev.   & Mean    & Std. Dev.   \\
\midrule
Bachelors or more &  & 0.6 & 0.5 & 0.4 & 0.5 & 0.5 & 0.5 & 0.4 & 0.5\\
\midrule
 &  & N & Pct. & N & Pct. & N & Pct. & N & Pct.\\
Age & 05-39 & 5762 & 35.6 & 1450 & 38.9 & 1356 & 33.7 & 8 & 0.1\\
 & 40-64 & 9570 & 59.1 & 2239 & 60.1 & 1705 & 42.3 & 1833 & 18.2\\
 & 65-79 & 836 & 5.2 & 33 & 0.9 & 903 & 22.4 & 6321 & 62.8\\
 & 80+ & 20 & 0.1 & 1 & 0.0 & 64 & 1.6 & 1898 & 18.9\\
Wheelchair & FALSE & 16161 & 99.8 & 3609 & 96.9 & 4010 & 99.6 & 9619 & 95.6\\
 & TRUE & 27 & 0.2 & 114 & 3.1 & 18 & 0.4 & 441 & 4.4\\
Income & < \$25,000 & 872 & 5.4 & 961 & 25.8 & 663 & 16.5 & 1825 & 18.1\\
 & \$25,000 - \$50,000 & 2235 & 13.8 & 632 & 17.0 & 713 & 17.7 & 2437 & 24.2\\
 & \$50,000 - \$100,000 & 5312 & 32.8 & 953 & 25.6 & 1203 & 29.9 & 3245 & 32.3\\
 & > \$100,000 & 7476 & 46.2 & 1102 & 29.6 & 1333 & 33.1 & 1975 & 19.6\\
Sex & Male & 8820 & 54.5 & 1192 & 32.0 & 1487 & 36.9 & 4488 & 44.6\\
 & Female & 7368 & 45.5 & 2531 & 68.0 & 2541 & 63.1 & 5572 & 55.4\\
 & I prefer not to answer & 0 & 0.0 & 0 & 0.0 & 0 & 0.0 & 0 & 0.0\\
 & I don't know & 0 & 0.0 & 0 & 0.0 & 0 & 0.0 & 0 & 0.0\\
Works from Home & -1 & 0 & 0.0 & 3721 & 99.9 & 267 & 6.6 & 10060 & 100.0\\
 & -7 & 2 & 0.0 & 0 & 0.0 & 1 & 0.0 & 0 & 0.0\\
 & -8 & 1 & 0.0 & 0 & 0.0 & 1 & 0.0 & 0 & 0.0\\
 & -9 & 728 & 4.5 & 0 & 0.0 & 0 & 0.0 & 0 & 0.0\\
 & 01 & 1718 & 10.6 & 0 & 0.0 & 940 & 23.3 & 0 & 0.0\\
 & 02 & 13739 & 84.9 & 2 & 0.1 & 2819 & 70.0 & 0 & 0.0\\
\bottomrule
\end{tabular}}
\end{table}
\end{landscape}

\hypertarget{results}{%
\section{Results}\label{results}}

\hypertarget{example-one-nhts}{%
\subsection{Example one NHTS}\label{example-one-nhts}}

We estimated the models using mlogit for R \citet{R}, \citet{mlogit}. As described above, the
alternatives for daily activity pattern choice are a Mandatory pattern where the
individual's day involves a work or school tour, a Non-Mandatory pattern where
only discretionary trips are taken, and a Home pattern where the individual does
not leave home. In the models estimated for this study, the Home pattern serves
as the reference alternative with a utility of zero. Retired and otherwise
non-working individuals choose only between Non-Mandatory and Home daily activity
patterns.

The model estimates are presented in
Table \citet{ref}(tab:model-coef).
The estimated coefficients are of the expected sign, though not all are significant.
Some predictors that proved to be insignificant, such as automobile availability
for full-time workers, were excluded from the estimated models.
The overall model fit --- as indicated by the McFadden \(\rho^2\) with respect to
a market shares (constants only) model --- is not strikingly high. Were the purpose
of this research to identify the best fit model of activity pattern choice for
each person segment we would undertake an exercise to include, exclude, and identify
potential transformations for different sets of variables. In this case, however,
the goal of these models is simply to provide a plausible comparison point for
the behavior of individuals using wheelchairs against the behavior of individuals
in other person type segments.

\begin{table}

\caption{\label{tab:model-coef}Daily Activity Pattern Model Estimates}
\centering
\resizebox{\linewidth}{!}{
\begin{tabular}[t]{llcccc}
\toprule
  &    & Full-time worker & Non-worker & Part-time worker & Retired\\
\midrule
(Intercept) & M & \num{2.083} (\num{15.207})*** &  & \num{1.545} (\num{10.300})*** & \\
 & NM & \num{1.137} (\num{7.854})*** & \num{0.591} (\num{6.359})*** & \num{0.338} (\num{2.088})* & \num{-1.169} (\num{-1.506})\\
wheelchairTRUE & M & \num{-1.851} (\num{-3.328})*** &  & \num{-3.315} (\num{-3.906})*** & \\
 & NM & \num{-0.625} (\num{-1.315}) & \num{-0.721} (\num{-3.647})*** & \num{-1.866} (\num{-3.560})*** & \num{-1.258} (\num{-11.924})***\\
male & M & \num{0.008} (\num{0.140}) &  & \num{-0.040} (\num{-0.347}) & \\
 & NM & \num{-0.148} (\num{-2.378})* & \num{-0.271} (\num{-3.477})*** & \num{-0.219} (\num{-1.828})+ & \num{0.235} (\num{4.798})***\\
bach\_degree & M & \num{0.353} (\num{5.716})*** &  & \num{0.360} (\num{3.033})** & \\
 & NM & \num{0.648} (\num{9.834})*** & \num{0.501} (\num{6.028})*** & \num{0.584} (\num{4.815})*** & \num{0.349} (\num{6.521})***\\
income\$25,000 - \$50,000 & M & \num{-0.055} (\num{-0.373}) &  & \num{0.169} (\num{0.876}) & \\
 & NM & \num{-0.371} (\num{-2.344})* & \num{-0.167} (\num{-1.506}) & \num{0.444} (\num{2.196})* & \num{-0.095} (\num{-1.358})\\
income\$50,000 - \$100,000 & M & \num{-0.175} (\num{-1.272}) &  & \num{-0.160} (\num{-0.971}) & \\
 & NM & \num{-0.312} (\num{-2.146})* & \num{-0.052} (\num{-0.506}) & \num{0.190} (\num{1.096}) & \num{0.115} (\num{1.643})\\
income> \$100,000 & M & \num{-0.206} (\num{-1.495}) &  & \num{-0.326} (\num{-1.994})* & \\
 & NM & \num{-0.233} (\num{-1.610}) & \num{-0.088} (\num{-0.844}) & \num{0.127} (\num{0.737}) & \num{0.036} (\num{0.444})\\
age\_bin40-64 & M & \num{-0.006} (\num{-0.104}) &  & \num{0.669} (\num{5.094})*** & \\
 & NM & \num{0.026} (\num{0.386}) & \num{0.433} (\num{5.788})*** & \num{1.055} (\num{7.805})*** & \num{2.241} (\num{2.886})**\\
age\_bin65-79 & M & \num{0.179} (\num{1.147}) &  & \num{0.395} (\num{2.614})** & \\
 & NM & \num{0.801} (\num{5.081})*** & \num{1.690} (\num{2.998})** & \num{0.721} (\num{4.647})*** & \num{2.132} (\num{2.752})**\\
age\_bin80+ & M & \num{17.592} (\num{0.004}) &  & \num{2.067} (\num{2.293})* & \\
 & NM & \num{17.247} (\num{0.004}) & \num{14.648} (\num{0.008}) & \num{2.154} (\num{2.391})* & \num{1.536} (\num{1.980})*\\
works\_home & M & \num{-1.542} (\num{-18.502})*** &  & \num{-1.340} (\num{-9.830})*** & \\
 & NM & \num{-0.044} (\num{-0.558}) &  & \num{0.112} (\num{0.869}) & \\
\midrule
 & Num.Obs. & \num{15895} & \num{3648} & \num{3912} & \num{9482}\\
 & AIC & \num{26550.2} & \num{4470.9} & \num{6965.2} & \num{10689.7}\\
 & BIC &  &  &  & \\
 & Log.Lik. & \num{-13253.089} & \num{-2225.474} & \num{-3460.598} & \num{-5334.858}\\
 & rho2 & \num{0.031} & \num{0.019} & \num{0.055} & \num{0.026}\\
 & rho20 & \num{0.241} & \num{0.120} & \num{0.195} & \num{0.188}\\
\bottomrule
\multicolumn{6}{l}{\rule{0pt}{1em}Coefficients represent utility change relative to H: stay at home pattern.}\\
\multicolumn{6}{l}{\rule{0pt}{1em}t-statistics in parentheses, * p $<$ 0.1, ** p $<$ 0.05, *** p $<$ 0.01}\\
\end{tabular}}
\end{table}

In this regard, the model results show strong divergence of the utility
preferences of individuals who use wheelchairs. For instance, full-time workers
in the middle income groups are modestly less likely to choose non-mandatory
patterns, and part-time workers of higher income are less like to choose
mandatory patterns. Income appears to have no discernible effect on the choices
of non-working and retired individuals. We see a negative utility score for all
person types with a wheelchair variable, and ``mandatory'' is even more negative.
This is expected as individuals with wheelchairs are less likely to take a work
or school trip compared to a shopping or a recreational trip. Non-workers and
retired person types do not have a coefficient for ``mandatory'' DAP because those
users by definition do not take ``mandatory'' DAP. Indeed, wheelchair use is among
the strongest predictors of daily activity pattern choice across population
segments.

\hypertarget{example-two-activitysim}{%
\subsection{Example two ActivitySim}\label{example-two-activitysim}}

As a secondary example, the research measures the impact of
wheelchair status on ActivitySim's selection of daily plans for our given
synthetic population. Given a ``Before'' scenario in ActivitySim of the Salt Lake
Area and ignoring the newly added wheelchair status in the synthetic population,
ActivitySim predicted a DAP for each individual. In a second, ``After'' scenario,
ActivitySim again predicted a DAP for each person, this time considering the
wheelchair use status of each individual in the population. We hypothesized that
those with wheelchairs and those in the same households as individuals with
wheelchairs would change their DAP because of the negative utility scores
applied to the ``mandatory'' and ``nonmandatory'' DAP alternatives, and the rest of
the population would be unaffected. The DAP of those within the same household
of a wheelchair user may change because of the coordinated nature of household
DAP in ActivitySim. Table 4-6 shows the change in DAP among those with
wheelchairs, in the same household as one with a wheelchair, and with neither a
wheelchair nor in the same household. The table contains both total volumes and
percentages; the value of percent is by total volume in the group, for example,
16.4 percent of Wheelchair Users chose a ``home'' pattern in both the ``Before''
scenario and the ``After'' scenario. The latter group is rightly unaffected by
the wheelchair implementation in the simulation (with the exception of a few
changes attributable to randomness) and does not include a percentage breakdown.
Primarily, DAP remain the same for most individuals, as shown in the diagonal.
However, there is a large volume of wheelchair users and their household members
that stay home, particularly from ``nonmandatory'' DAP. This finding is consistent
with our hypothesis.

\begin{table}

\caption{\label{tab:dap-summary}Daily Activity Pattern Change}
\centering
\begin{tabular}[t]{llrrr}
\toprule
\multicolumn{1}{c}{} & \multicolumn{1}{c}{} & \multicolumn{3}{c}{DAP with Wheelchair Use} \\
\cmidrule(l{3pt}r{3pt}){3-5}
Group & DAP without Wheelchair Use & H & M & N\\
\midrule
 & H & 3369 & 20 & 459\\

 & M & 932 & 1642 & 308\\

\multirow{-3}{*}{\raggedright\arraybackslash Wheelchair Users} & N & 3584 & 23 & 10261\\
\cmidrule{1-5}
 & H & 4511 & 213 & 631\\

 & M & 759 & 15409 & 301\\

\multirow{-3}{*}{\raggedright\arraybackslash Household Members} & N & 1235 & 415 & 13119\\
\cmidrule{1-5}
 & H & 309965 & 2 & \\

 & M & 2 & 1460582 & \\

\multirow{-3}{*}{\raggedright\arraybackslash Not Affected} & N &  & 2 & 659258\\
\bottomrule
\end{tabular}
\end{table}

\hypertarget{section}{%
\section{}\label{section}}

We have finished a nice book.

\hypertarget{conclusion}{%
\section{Conclusion}\label{conclusion}}

This is the end of the paper.

\hypertarget{acks}{%
\section*{Acknowledgements}\label{acks}}
\addcontentsline{toc}{section}{Acknowledgements}

Figures and tables in this paper were created with a variety of R
packages.

\bibliography{book.bib}


\end{document}
