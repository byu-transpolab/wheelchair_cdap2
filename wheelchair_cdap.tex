\documentclass[]{elsarticle} %review=doublespace preprint=single 5p=2 column
%%% Begin My package additions %%%%%%%%%%%%%%%%%%%
\usepackage[hyphens]{url}

  \journal{Submitted to Journal} % Sets Journal name


\usepackage{lineno} % add
\providecommand{\tightlist}{%
  \setlength{\itemsep}{0pt}\setlength{\parskip}{0pt}}

\usepackage{graphicx}
\usepackage{booktabs} % book-quality tables
%%%%%%%%%%%%%%%% end my additions to header

\usepackage[T1]{fontenc}
\usepackage{lmodern}
\usepackage{amssymb,amsmath}
\usepackage{ifxetex,ifluatex}
\usepackage{fixltx2e} % provides \textsubscript
% use upquote if available, for straight quotes in verbatim environments
\IfFileExists{upquote.sty}{\usepackage{upquote}}{}
\ifnum 0\ifxetex 1\fi\ifluatex 1\fi=0 % if pdftex
  \usepackage[utf8]{inputenc}
\else % if luatex or xelatex
  \usepackage{fontspec}
  \ifxetex
    \usepackage{xltxtra,xunicode}
  \fi
  \defaultfontfeatures{Mapping=tex-text,Scale=MatchLowercase}
  \newcommand{\euro}{€}
\fi
% use microtype if available
\IfFileExists{microtype.sty}{\usepackage{microtype}}{}
\usepackage{natbib}
\bibliographystyle{plainnat}
\usepackage{color}
\usepackage{fancyvrb}
\newcommand{\VerbBar}{|}
\newcommand{\VERB}{\Verb[commandchars=\\\{\}]}
\DefineVerbatimEnvironment{Highlighting}{Verbatim}{commandchars=\\\{\}}
% Add ',fontsize=\small' for more characters per line
\usepackage{framed}
\definecolor{shadecolor}{RGB}{248,248,248}
\newenvironment{Shaded}{\begin{snugshade}}{\end{snugshade}}
\newcommand{\AlertTok}[1]{\textcolor[rgb]{0.94,0.16,0.16}{#1}}
\newcommand{\AnnotationTok}[1]{\textcolor[rgb]{0.56,0.35,0.01}{\textbf{\textit{#1}}}}
\newcommand{\AttributeTok}[1]{\textcolor[rgb]{0.77,0.63,0.00}{#1}}
\newcommand{\BaseNTok}[1]{\textcolor[rgb]{0.00,0.00,0.81}{#1}}
\newcommand{\BuiltInTok}[1]{#1}
\newcommand{\CharTok}[1]{\textcolor[rgb]{0.31,0.60,0.02}{#1}}
\newcommand{\CommentTok}[1]{\textcolor[rgb]{0.56,0.35,0.01}{\textit{#1}}}
\newcommand{\CommentVarTok}[1]{\textcolor[rgb]{0.56,0.35,0.01}{\textbf{\textit{#1}}}}
\newcommand{\ConstantTok}[1]{\textcolor[rgb]{0.00,0.00,0.00}{#1}}
\newcommand{\ControlFlowTok}[1]{\textcolor[rgb]{0.13,0.29,0.53}{\textbf{#1}}}
\newcommand{\DataTypeTok}[1]{\textcolor[rgb]{0.13,0.29,0.53}{#1}}
\newcommand{\DecValTok}[1]{\textcolor[rgb]{0.00,0.00,0.81}{#1}}
\newcommand{\DocumentationTok}[1]{\textcolor[rgb]{0.56,0.35,0.01}{\textbf{\textit{#1}}}}
\newcommand{\ErrorTok}[1]{\textcolor[rgb]{0.64,0.00,0.00}{\textbf{#1}}}
\newcommand{\ExtensionTok}[1]{#1}
\newcommand{\FloatTok}[1]{\textcolor[rgb]{0.00,0.00,0.81}{#1}}
\newcommand{\FunctionTok}[1]{\textcolor[rgb]{0.00,0.00,0.00}{#1}}
\newcommand{\ImportTok}[1]{#1}
\newcommand{\InformationTok}[1]{\textcolor[rgb]{0.56,0.35,0.01}{\textbf{\textit{#1}}}}
\newcommand{\KeywordTok}[1]{\textcolor[rgb]{0.13,0.29,0.53}{\textbf{#1}}}
\newcommand{\NormalTok}[1]{#1}
\newcommand{\OperatorTok}[1]{\textcolor[rgb]{0.81,0.36,0.00}{\textbf{#1}}}
\newcommand{\OtherTok}[1]{\textcolor[rgb]{0.56,0.35,0.01}{#1}}
\newcommand{\PreprocessorTok}[1]{\textcolor[rgb]{0.56,0.35,0.01}{\textit{#1}}}
\newcommand{\RegionMarkerTok}[1]{#1}
\newcommand{\SpecialCharTok}[1]{\textcolor[rgb]{0.00,0.00,0.00}{#1}}
\newcommand{\SpecialStringTok}[1]{\textcolor[rgb]{0.31,0.60,0.02}{#1}}
\newcommand{\StringTok}[1]{\textcolor[rgb]{0.31,0.60,0.02}{#1}}
\newcommand{\VariableTok}[1]{\textcolor[rgb]{0.00,0.00,0.00}{#1}}
\newcommand{\VerbatimStringTok}[1]{\textcolor[rgb]{0.31,0.60,0.02}{#1}}
\newcommand{\WarningTok}[1]{\textcolor[rgb]{0.56,0.35,0.01}{\textbf{\textit{#1}}}}
\usepackage{longtable}
\ifxetex
  \usepackage[setpagesize=false, % page size defined by xetex
              unicode=false, % unicode breaks when used with xetex
              xetex]{hyperref}
\else
  \usepackage[unicode=true]{hyperref}
\fi
\hypersetup{breaklinks=true,
            bookmarks=true,
            pdfauthor={},
            pdftitle={Coordinated daily activity patterns of wheelchair users.},
            colorlinks=false,
            urlcolor=blue,
            linkcolor=magenta,
            pdfborder={0 0 0}}
\urlstyle{same}  % don't use monospace font for urls

\setcounter{secnumdepth}{5}
% Pandoc toggle for numbering sections (defaults to be off)

% Pandoc citation processing

% Pandoc header
\usepackage{booktabs}
\usepackage{booktabs}
\usepackage{longtable}
\usepackage{array}
\usepackage{multirow}
\usepackage{wrapfig}
\usepackage{float}
\usepackage{colortbl}
\usepackage{pdflscape}
\usepackage{tabu}
\usepackage{threeparttable}
\usepackage{threeparttablex}
\usepackage[normalem]{ulem}
\usepackage{makecell}
\usepackage{xcolor}



\begin{document}
\begin{frontmatter}

  \title{Coordinated daily activity patterns of wheelchair users.}
    \author[Brigham Young University]{Gregory Macfarlane\corref{1}}
   \ead{gregmacfarlane@byu.edu} 
    \author[Brigham Young University]{Nate Lant}
   \ead{natelant@gmail.com} 
      \address[Brigham Young University]{Civil and Environmental Engineering Department, 430 Engineering Building, Provo, Utah 84602}
      \cortext[1]{Corresponding Author}
  
  \begin{abstract}
  Individuals who use wheelchairs or who have other mobility challenges are sometimes unable to effectively access modern mobility as a service (MaaS) systems including application-based ridehailing and on-demand microtransit, etc. Even designing a MaaS targeted at these users is challenging, given the limited prior analysis of their travel and activity patterns. In this research, we present an initial attempt to model the daily activity pattern choice of respondents to the 2017 National Household Travel Survey who self-identify as using a wheelchair. We find that wheelchair use is a significant predictor of activity patterns, with individuals who use wheelchairs considerably less likely to choose out-of-home daily patterns. We additionally find that including wheelchair use as a variable in a model with a common person-type segmentation is statistically preferable to specifying wheelchair users as an independent segment. Further analysis of household-level coordination of daily activity patterns using more detailed surveys is warranted.
  \end{abstract}
  
 \end{frontmatter}

\hypertarget{intro}{%
\section{Introduction}\label{intro}}

\hypertarget{literature}{%
\section{Literature}\label{literature}}

Here is a review of existing methods.

\hypertarget{methods}{%
\section{Methods}\label{methods}}

We describe our methods in this chapter.

\hypertarget{data}{%
\subsection{Data}\label{data}}

\hypertarget{applications}{%
\section{Applications}\label{applications}}

Some \emph{significant} applications are demonstrated in this chapter.

\hypertarget{example-one}{%
\subsection{Example one}\label{example-one}}

\begin{Shaded}
\begin{Highlighting}[]
\FunctionTok{library}\NormalTok{(targets)}
\FunctionTok{library}\NormalTok{(kableExtra)}
\end{Highlighting}
\end{Shaded}

\begin{Shaded}
\begin{Highlighting}[]
\FunctionTok{tar\_load}\NormalTok{(dap\_table)}
\FunctionTok{kbl}\NormalTok{(dap\_table)}
\end{Highlighting}
\end{Shaded}

\begin{tabular}[t]{l|l|r|r|r}
\hline
Group & DAP\_before & H & M & N\\
\hline
Wheelchair Users & H & 3369 & 20 & 459\\
\hline
Wheelchair Users & M & 932 & 1642 & 308\\
\hline
Wheelchair Users & N & 3584 & 23 & 10261\\
\hline
Household Members & H & 4511 & 213 & 631\\
\hline
Household Members & M & 759 & 15409 & 301\\
\hline
Household Members & N & 1235 & 415 & 13119\\
\hline
Not Affected & H & 309965 & 2 & NA\\
\hline
Not Affected & M & 2 & 1460582 & NA\\
\hline
Not Affected & N & NA & 2 & 659258\\
\hline
\end{tabular}

\hypertarget{example-two}{%
\subsection{Example two}\label{example-two}}

\hypertarget{final-words}{%
\section{Final Words}\label{final-words}}

We have finished a nice book.

\bibliography{book.bib}


\end{document}

